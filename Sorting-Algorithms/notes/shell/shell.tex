\documentclass{article}
\usepackage{amsmath}
\usepackage{graphicx}

\title{Shell Sort}
\author{Sandip Sapkota}
\date{\today}

\begin{document}

\maketitle

\section*{Introduction}
Shell Sort is an in-place comparison-based sorting algorithm. It can be seen as a generalization of insertion sort that allows the exchange of items that are far apart. The idea is to arrange the list of elements so that, starting anywhere, taking every $h$th element produces a sorted list. Such a list is said to be $h$-sorted. The algorithm uses a sequence of increments $h_1, h_2, \ldots, h_k$, called the gap sequence.

\section*{Pseudo Code}
The following is the pseudo code for Shell Sort:

\begin{verbatim}
function shellSort(arr)
  n = length(arr)
  gap = n // 2
  while gap > 0
    for i = gap to n - 1
      temp = arr[i]
      j = i
      while j >= gap and arr[j - gap] > temp
        arr[j] = arr[j - gap]
        j = j - gap
      arr[j] = temp
    gap = gap // 2
end function
\end{verbatim}

\section*{Example}
Consider the array: [12, 34, 54, 2, 3]

\begin{enumerate}
  \item Initial array: [12, 34, 54, 2, 3]
  \item Using gap sequence: [5, 2, 1]
  \item First pass with gap = 2: [12, 34, 54, 2, 3] $\rightarrow$ [12, 3, 54, 2, 34]
  \item Second pass with gap = 1: [12, 3, 54, 2, 34] $\rightarrow$ [3, 12, 2, 34, 54] $\rightarrow$ [3, 2, 12, 34, 54]
  \item Final sorted array: [2, 3, 12, 34, 54]
\end{enumerate}

\section*{Time Complexity}
The time complexity of Shell Sort depends on the gap sequence used. The best known sequence is the one proposed by Donald Shell, which has a time complexity of $O(n^2)$.


\end{document}