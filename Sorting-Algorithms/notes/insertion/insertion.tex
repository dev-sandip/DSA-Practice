\documentclass{article}
\usepackage{amsmath}

\title{Insertion Sort}
\author{Sandip Sapkota}
\date{\today}

\begin{document}

\maketitle

\section*{Introduction}

Insertion sort is a simple sorting algorithm that builds the final sorted array (or list) one item at a time. It is much less efficient on large lists than more advanced algorithms such as quicksort, heapsort, or merge sort.\\

\textbf{ Time Complexity:} $O(n^2)$
\section*{Algorithm}

In insertion sort, we basically sort the data by holding it in another array. In this algorithm, we hold the data in a temporary variable (let's call it \texttt{hold}) at the \texttt{i}th index. Then we set \texttt{j} as \texttt{i-1}. Using a while loop, we compare the \texttt{hold} data with the data in the array. If the data is greater than the \texttt{hold} data, we shift the data to the right side and decrement \texttt{j} by 1. If the data is smaller than the \texttt{hold} data, we insert the \texttt{hold} data at \texttt{j+1} index.

\section*{Pseudocode}

\begin{verbatim}
for i = 1 to length(A) - 1
  hold = A[i]
  j = i - 1
  while j >= 0 and A[j] > hold
    A[j + 1] = A[j]
    j = j - 1
  A[j + 1] = hold
  \end{verbatim}

\section*{Example}
Consider the array \{5, 2, 4, 6, 1, 3\}. The insertion sort algorithm processes the array as follows:

\begin{itemize}
  \item Initially: \{5\textbf{!}, 2, 4, 6, 1, 3\}
  \item After first iteration: \{2, 5\textbf{!}, 4, 6, 1, 3\}
  \item After second iteration: \{2, 4, 5\textbf{!}, 6, 1, 3\}
  \item After third iteration: \{2, 4, 5, 6\textbf{!}, 1, 3\}
  \item After fourth iteration: \{1, 2, 4, 5, 6\textbf{!}, 3\}
  \item After fifth iteration: \{1, 2, 3, 4, 5, 6\textbf{!}\}
\end{itemize}
Note: Here \textbf{!} denotes the position of the \texttt{hold} data.
\end{document}