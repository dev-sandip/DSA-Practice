\documentclass{article}
\usepackage{forest}

\title{DSA Practice: Trees}
\author{Sandip}
\date{\today}

\begin{document}

\maketitle
This document contains past year questions for the chapter \textbf{Trees} in Data Structures and Algorithms.

\newpage
\tableofcontents

\newpage
\section{Questions}

\subsection{2080 Chaitra}
\subsubsection{Question 1}
\textbf{What do you mean by a complete binary tree? Explain with an example. Create an AVL Balanced tree for the following sequence of elements: 10, 9, 8, 4, 5, 7, 32, 16, 11, 1, 12, and 2.}

A complete binary tree is a binary tree in which all levels are completely filled, except possibly the last level, which is filled from left to right.

\bigskip
\textbf{Example of a Complete Binary Tree:}

\begin{center}
\begin{forest}
for tree={
    grow=south,
    circle, draw, minimum size=3ex, inner sep=1pt,
    s sep=5mm
}
[1
    [2
        [4]
        [5]
    ]
    [3
        [6]
        [7]
    ]
]
\end{forest}
\end{center}

In the above example, the binary tree is complete because:
- All levels are completely filled except possibly the last.
- The last level is filled from left to right.


\subsubsection{AVL Balanced Tree}
An AVL tree is a self-balancing binary search tree where the difference between the heights of left and right subtrees cannot be more than one.

For the given sequence:  
10, 9, 8, 4, 5, 7, 32, 16, 11, 1, 12, 2  

\bigskip
\textbf{Step 1: Binary Search Tree (Before Balancing)}
\begin{center}
\begin{forest}
for tree={
    grow=south,
    circle, draw, minimum size=3ex, inner sep=1pt,
    s sep=5mm
}
[10
    [9
        [8
            [4
                [1
                    [,no edge, draw=none]
                    [2]
                ]
                [5
                    [,no edge, draw=none]
                    [7]
                ]
            ]
            [,no edge , draw=none]
        ]
    ]
    [32
        [16
            [11
                [,no edge, draw=none]
                [12]
            ]
            [,no edge , draw=none]
        ]

    ]
    
]
\end{forest}
\end{center}

\bigskip
\textbf{Step 2: AVL Balanced Tree (After Rotations)}
\begin{center}
\begin{forest}
for tree={
    grow=south,
    circle, draw, minimum size=3ex, inner sep=1pt,
    s sep=5mm
}
[8
    [5
        [2
            [1]
            [4]
        ]
        [7]
    ]
    [11
        [10[9][,no edge, draw=none]]
        [16
            [12]
            [32
                [,no edge, draw=none]
            ]
        ]
    ]
]
\end{forest}

\end{center}

\end{document}
